\documentclass[a4paper,12pt]{article}
\usepackage[utf8]{inputenc}
\usepackage{listings}
\usepackage{xcolor}

\title{Implementación del Analizador Léxico}
\author{Universidad La Salle \\ Escuela Profesional de Ingeniería de Software}
\date{\today}

\begin{document}

\maketitle

\section{Introducción}
Este documento describe la implementación de un analizador léxico utilizando la biblioteca \texttt{Ply} de Python. El analizador está diseñado para reconocer tokens específicos de un lenguaje propuesto.

El análisis léxico es una fase fundamental en la construcción de compiladores, encargándose de transformar la entrada de texto en una secuencia de tokens. Estos tokens representan unidades léxicas reconocibles del lenguaje, como palabras clave, operadores y símbolos especiales.

\section{Especificación Léxica}
Los tokens que reconoce el analizador son:
\begin{itemize}
    \item Variables, constantes y tipos de datos como \texttt{DT\_INT}, \texttt{DT\_FLOAT}, \texttt{DT\_BOOL}, entre otros.
    \item Estructuras de control como \texttt{IF}, \texttt{ELSE}, \texttt{SWITCH}, \texttt{FOR}, \texttt{WHILE}, etc.
    \item Operadores aritméticos como \texttt{+}, \texttt{-}, \texttt{*}, \texttt{/} y \texttt{\%}.
    \item Operadores de flujo de entrada/salida como \texttt{COUT}, \texttt{CIN}, \texttt{O\_IN} y \texttt{O\_EX}.
    \item Operadores relacionales como \texttt{==}, \texttt{!=}, \texttt{>}, \texttt{<}, \texttt{>=}, \texttt{<=}.
    \item Símbolos especiales como paréntesis, llaves y punto y coma.
\end{itemize}

\section{Código Fuente}
A continuación se presenta la implementación del analizador léxico en Python:

\begin{lstlisting}[language=Python, caption=Código del analizador léxico, basicstyle=\ttfamily, keywordstyle=\color{blue}]
import ply.lex as lex

tokens = (
    'DT_STRING',
    'VAR_KEYWORD', 'VAR', 'CONST', 'DT_INT', 'DT_FLOAT', 'DT_CHAR', 'DT_BOOL',
    'IF', 'ELSE', 'SWITCH', 'FOR', 'WHILE', 'DO_WHILE',
    'RETURN', 'BREAK', 'CONTINUE',
    'COUT', 'CIN', 'O_IN', 'O_EX', 'ENDLINE',
    'O_SUMA', 'O_REST', 'O_MULT', 'O_DIV', 'O_MOD',
    'O_IGUALDAD', 'O_DIFERENTE', 'O_MAYOR', 'O_MENOR', 'O_MAYORIGUAL', 'O_ASIG',
    'SYM_PAR_IZQ', 'SYM_PAR_DER', 'SYM_LLAVE_IZQ', 'SYM_LLAVE_DER',
    'FUNCTION', 'E_INS', 'COM_SIMPLE', 'COM_MULTI'
)

t_O_SUMA = r'\\+'
t_O_REST = r'-'
t_O_MULT = r'\\*'
t_O_DIV = r'/'
t_O_MOD = r'%'
t_O_IGUALDAD = r'=='
t_O_DIFERENTE = r'!='
t_O_MAYOR = r'>'
t_O_MENOR = r'<'
t_O_MAYORIGUAL = r'>='
t_O_ASIG = r'='
t_SYM_PAR_IZQ = r'\\('
t_SYM_PAR_DER = r'\\)'
t_SYM_LLAVE_IZQ = r'\\{'
t_SYM_LLAVE_DER = r'\\}'
t_O_IN = r'<<'
t_O_EX = r'>>'
t_ENDLINE = r'ENDLINE'
t_E_INS = r';'

def t_VAR_KEYWORD(t):
    r'VAR(?![a-zA-Z0-9-])'
    return t

def t_VAR(t):
    r'V-[a-zA-Z]+-[0-9]+'
    return t

def t_DT_INT(t):
    r'(DT-INT|[0-9]+)'
    if t.value.isdigit():
        t.value = int(t.value)
    return t

def t_DT_STRING(t):
    r'\".+\"'
    return t

lexer = lex.lex()
\end{lstlisting}

\section{Pruebas y Resultados}
Para verificar la funcionalidad del analizador léxico, se realizaron pruebas con distintos fragmentos de código de entrada. Se analizaron los siguientes casos:
\begin{enumerate}
    \item Un código simple de "Hola Mundo".
    \item Un código que usa estructuras de control como condicionales y bucles.
    \item Un código que define y utiliza funciones.
\end{enumerate}

Cada uno de estos casos produjo una lista de tokens correctamente reconocidos por el analizador. Los resultados obtenidos muestran que el analizador identifica correctamente los tokens definidos y distingue entre palabras clave, identificadores y operadores.

\section{Conclusión}
El uso de \texttt{Ply} facilita la implementación de un analizador léxico eficiente y modular. Se reconoce correctamente la sintaxis del lenguaje propuesto, permitiendo la identificación de distintos tipos de tokens. 

Esta implementación sienta las bases para futuras mejoras en la creación de compiladores más avanzados, integrando la fase de análisis sintáctico y semántico.

\end{document}